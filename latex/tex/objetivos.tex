%%%%%%%%%%%%%%%%%%%%%%%%%%%%%%%%%%%%%%%%%%%%%%%%%%%%%%%%
%%%%%%%%%%%%%%%%%%%%%%%%%%%%%%%%%%%%%%%%%%%%%%%%%%%%%%%%
%%%%  El informe no podrá exceder las 25 páginas   %%%%%
%%%%%%%%%%%%%%%%%%%%%%%%%%%%%%%%%%%%%%%%%%%%%%%%%%%%%%%%
%%%%%%%%%%%%%%%%%%%%%%%%%%%%%%%%%%%%%%%%%%%%%%%%%%%%%%%%

\section{Motivación}
\label{sec:motivacion}

	El informe del anteproyecto contó con la documentación de las especificaciones técnicas y funcionales con las que debía contar el dispositivo. Una vez aprobado dicho anteproyecto, se llevó a cabo la construcción del prototipo.
    
    A lo largo de la construcción del prototipo, se encontraron algunos obstáculos funcionales, los cuales requirieron realizar pequeñas modificaciones en el circuito. Estas modificaciones tuvieron el propósito de enmendar algunos comportamientos del circuito que no habían sido predichos en el diseño del anteproyecto.
    
    Finalmente, luego de tener el prototipo en funcionamiento, llegó el momento de realizar la validación del mismo.
    
\section{Objetivos}

	Los objetivos de este trabajo práctico son
    
\begin{itemize}
\item Documentar los cambios realizados sobre el diseño del circuito presentado en el anteproyecto.
\item Documentar el diseño último de las partes faltantes en el anteproyecto. Estas son el diseño del PCB y el diseño del software que otorga la funcionalidad esperada del dispositivo.
\item Documentar las mediciones hechas sobre el dispositivo, que validan su funcionamiento tal como se documentó en las especificaciones del anteproyecto. 
\end{itemize}


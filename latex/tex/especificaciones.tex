%%%%%%%%%%%%%%%%%%%%%%%%%%%%%%%%%%%%%%%%%%%%%%%%%%%%%%%%
%%%%%%%%%%%%%%%%%%%%%%%%%%%%%%%%%%%%%%%%%%%%%%%%%%%%%%%%
%%%%  El informe no podrá exceder las 25 páginas   %%%%%
%%%%%%%%%%%%%%%%%%%%%%%%%%%%%%%%%%%%%%%%%%%%%%%%%%%%%%%%
%%%%%%%%%%%%%%%%%%%%%%%%%%%%%%%%%%%%%%%%%%%%%%%%%%%%%%%%
\section{Resumen de los requerimientos\label{sec:Requerimientos}}

 Para poder realizar un mejor análisis de la documentación de las mediciones que se detallarán más adelante, se relevan los requerimientos expuestos en el anteproyecto del dispositivo.

  \subsection{Requerimientos del usuario}

 \begin{itemize}

% \item Conocimiento de las capacidades del dispositivo.%no lo entendí

 \item Que el dispositivo tenga un manejo intuitivo.%Comando intuitivo del dispositivo.

% \item Conocimiento del estado actual del dispositivo. %no lo entendí
 \item Que el control de potencia responda en tiempo real.
 \item Que la comunicación sea transparente al usuario.
 \item Que la comunicación no interfiera con otros dispositivos, conectados o no a la red eléctrica.

 \end{itemize}

 \subsection{Requerimientos técnicos}

 \begin{itemize}

 \item Velocidad de comunicación: la tasa de transferencia de datos debe permitir un control que aparente ser de tiempo real para el usuario.

 \item Protocolo de comunicación: el protocolo de comunicación debe soportar una comunicación simple en el medio previsto, de baja tasa de transferencia pero con una buena confiabilidad.

 \item Distancia máxima de comunicación: la distancia máxima entre emisor y receptor será del orden de las distancias máximas entre tomas en una instalación hogareña o en un taller o pequeña industria. 

 \item Potencia máxima de salida: la carga máxima a controlar será una carga típica encontrada en el hogar o en la industria liviana, como motores y artefactos comunes.
 
 \item Compatibilidad con la red eléctrica argentina de baja tensión.
 
 \item Aislación mecánica para mejorar la seguridad eléctrica.

\item Funcionamiento del circuito conectado en fase o en contrafase a la red indistintamente


 \end{itemize}

\section{Especificaciones}
Éstas son las especificaciones del dispositivo que fueron documentados en el anteproyecto del dispositivo.

% \subsection{Especificaciones de usuario}

%  \begin{itemize}

%  \item El transmisor se controlará con un potenciometro con la finalidad de poder regular la carga a distancia
 
%  \item Se espera una rápida respuesta del controlador.
 
%  \item El objetivo es que tanto el transmisor como el receptor se puedan conectar al toma-corriente de la instalación.


%  \end{itemize}

\subsection{Especificaciones técnicas funcionales} \label{ETF}
Se esperan cumplir o superar las siguientes especificaciones técnicas: 
\begin{center}
\begin{tabular}{| c | c |}
      	\hline
		Parámetro & Valor \\
        \hline
        \hline
        Mínima distancia de conexionado & 5 metros \\
        Máximo retardo en la aplicación del comando & 1 segundo \\
        Mínima cantidad de estados dimerizados & 4 estados \\
        Máxima amplitud de tensión inyectada & 5 V \\
        
        \hline
\end{tabular}
\end{center}

\subsection{Especificaciones de diseño circuital} 

Las especificaciones de diseño circuital que se deben cumplir para cubrir los requerimientos técnicos son las siguientes:

\begin{center}
\begin{tabular}{| c | c |}
      	\hline
		Parámetro & Valor \\
        \hline
        \hline        
        Potencia de control en la carga & 600 W \\
        Tensión de línea de trabajo & 220 V \\
        Frecuencia de línea de trabajo & 50 Hz \\
        % Rellenen como se les plazca
        
                
        \hline
\end{tabular}
\end{center}
% -PARA CONSIDERAR EL PROTOTIPO DEL PROYECTO APROBADO

% 1. Al menos una etapa del circuito deberá estar implementada en tecnología SMD.

% 2. Deberá ser completamente caracterizable.

% 3. Deberá cumplir las especificaciones propuestas.

% 4. Deberá estar encuadrado en el proyecto propuesto originalmente.

% 5. Haber considerado la calidad técnica y las condiciones ambientales que enfrentará (Ej. MIL-STD-810, IRAM2392, IRAM4025, etc.)


% DISEÑO DE CIRCUITOS ELECTRÓNICOS EN 5 PASOS

%     Se determinan las especificaciones de diseño. En la mayoría de los casos se dispone de requerimientos del usuario, requerimientos técnicos, y consideraciones de mantenibilidad, confiablidad y manufacturabilidad.
%     Se analizan diferentes alternativas de diseño. Se desarrollan los diagramas en bloques hasta que cada bloque posea una única función.
%     Se analizan posibles soluciones circuitales para cada bloque y se selecciona una solución circuital, según los criterios tecnológicos y económicos adoptados.
%     Se calculan y seleccionan todos los componentes. Se realizan simulaciones y mediciones de las etapas individuales y sus interacciones. Se optimiza el diseño.
%     Se construye un prototipo de producción y se realizan todas las pruebas funcionales y ambientales.

% CRITERIOS PARA LA EVALUACIÓN DE DISEÑO DE CIRCUITOS
%  ELECTRÓNICOS 
% DISEÑO CONCEPTUAL 
% 1.Comprende el problema a resolver 
% 2.Análiza los requerimientos del usuario 
% 3.Define los requerimientos técnicos 
% 4.Define y Analiza  las especificaciones funcionale
% s y de diseño* 
% 5.Releva y Analiza soluciones existentes 
% 6.Propone alternativas de diseño y selecciona una s
% olución adecuada 
% 7.Tiene en cuenta  qué mediciones  deberán realizar
% se, cómo se realizarán y los recursos necesarios 
% 8.Plantea los diagramas en bloques del sistema y su
% b-sistemas* 
% 9.Explica y Comprende el funcionamiento del sistema
%  y sub-sistemas* 
% DISEÑO CIRCUITAL 
% 1.Explora distintos circuitos y analiza CORRECTAMEN
% TE su funcionamiento* 
% 2.Calcula CORRECTAMENTE TODOS los componentes de ci
% rcuitos individuales y las condiciones de funcionam
% iento* 
% 3.Investiga y selecciona los componentes 
% 4.Valida y optimiza el diseño mediante simulaciones
%  y mediciones, y determina todos los parámetros de 
% funcionamiento de los circuitos 
% 5.Determina si las especificaciones del circuito so
% n alcanzables* 
% 6.Realiza las simulaciones, indica y explica los ci
% rcuitos simulados, los puntos de medición, los pará
% metros utilizados y los resultados 
% obtenidos 
% 7.Realiza mediciones, indica y explica los circuito
% s implementados, las mediciones realizadas, los ins
% trumentos utilizados y los 
% resultados obtenidos* 
% 8.Realiza los diagramas esquemáticos con las refere
% ncias de todos sus componentes 
% 9.Realiza el listado de componentes indicando refer
% encia, descripción, valor, parámetros, fabricantes 
% y posibles proveedores 
% para cada 
% componente 
%
% INTEGRACIÓN 
% 1. Analiza mínimamente los condicionantes eléctrico
% s (SE y CEM), mecánicos (vibraciones y rigidez) y t
% érmicos (disipación de los 
% componentes)* 
% 2. Diseña los circuitos impresos de acuerdo a las r
% eglas básicas de ruteo* 
% 3. Dimensiona el montaje de los distintos módulos, 
% los puntos de fijación, los mecanismos de disipació
% n y el conexionado 
% 4. Muestra en detalle la localización de los compon
% entes, el diagrama de conexionado y el ensamblado d
% el prototipo 
% 5. Presenta el diagrama esquemático COMPLETO, lista
% do de componentes y partes COMPLETO y el listado de
%  proveedores* 
% 6. Realiza pruebas funcionales y ambientales básica
% s, indicando el procedimiento de ensayo y los resul
% tados obtenidos 
% 7. Analiza los modos y efectos de falla de cada com
% ponente 
% 8. Determina la confiabilidad de los componentes  
% 9. Optimiza el diseño en base a los resultados de l
% os puntos 6, 7 y 8 
% MEDICIONES Y OPTIMIZACIÓN 
% 1. Diseña un plan de ensayos funcionales y ambienta
% les* 
% 2. Diseña un plan de ajustes y verificaciones para 
% la puesta en marcha* 
% 3. Realiza correctamente las mediciones de todos lo
% s parámetros de funcionamiento * 
% 4. Presenta los instrumentos utilizados, los bancos
%  de medición, procedimientos y los resultados 
% 5. Analiza los resultados de las mediciones 
% 6. Compara las especificaciones técnicas con los pa
% rámetros caracterizados de acuerdo al plan de ensay
% os 
% 7. Compara valores calculados, simulados y medidos 
% 8. Analiza los objetivos alcanzados y hace recomend
% aciones para futuros diseños 
% 9. Optimiza el diseño en base a los resultados de l
% as mediciones para cumplir con las especificacione
%%%%%%%%%%%%%%%%%%%%%%%%%%%%%%%%%%%%%%%%%%%%%%%%%%%%%%%%
%%%%%%%%%%%%%%%%%%%%%%%%%%%%%%%%%%%%%%%%%%%%%%%%%%%%%%%%
%%%%  El informe no podrá exceder las 25 páginas   %%%%%
%%%%%%%%%%%%%%%%%%%%%%%%%%%%%%%%%%%%%%%%%%%%%%%%%%%%%%%%
%%%%%%%%%%%%%%%%%%%%%%%%%%%%%%%%%%%%%%%%%%%%%%%%%%%%%%%%
